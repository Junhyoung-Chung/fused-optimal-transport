\documentclass{article}

% if you need to pass options to natbib, use, e.g.:
%     \PassOptionsToPackage{numbers, compress}{natbib}
% before loading neurips_2025

% The authors should use one of these tracks.
% Before accepting by the NeurIPS conference, select one of the options below.
% 0. "default" for submission
% \usepackage{neurips_2025}
% the "default" option is equal to the "main" option, which is used for the Main Track with double-blind reviewing.
% 1. "main" option is used for the Main Track
%  \usepackage[main]{neurips_2025}
% 2. "position" option is used for the Position Paper Track
%  \usepackage[position]{neurips_2025}
% 3. "dandb" option is used for the Datasets & Benchmarks Track
 % \usepackage[dandb]{neurips_2025}
% 4. "creativeai" option is used for the Creative AI Track
%  \usepackage[creativeai]{neurips_2025}
% 5. "sglblindworkshop" option is used for the Workshop with single-blind reviewing
 % \usepackage[sglblindworkshop]{neurips_2025}
% 6. "dblblindworkshop" option is used for the Workshop with double-blind reviewing
%  \usepackage[dblblindworkshop]{neurips_2025}

% After being accepted, the authors should add "final" behind the track to compile a camera-ready version.
% 1. Main Track
\usepackage[main, final]{neurips_2025}
% 2. Position Paper Track
%  \usepackage[position, final]{neurips_2025}
% 3. Datasets & Benchmarks Track
 % \usepackage[dandb, final]{neurips_2025}
% 4. Creative AI Track
%  \usepackage[creativeai, final]{neurips_2025}
% 5. Workshop with single-blind reviewing
%  \usepackage[sglblindworkshop, final]{neurips_2025}
% 6. Workshop with double-blind reviewing
%  \usepackage[dblblindworkshop, final]{neurips_2025}
% Note. For the workshop paper template, both \title{} and \workshoptitle{} are required, with the former indicating the paper title shown in the title and the latter indicating the workshop title displayed in the footnote.
% For workshops (5., 6.), the authors should add the name of the workshop, "\workshoptitle" command is used to set the workshop title.
% \workshoptitle{WORKSHOP TITLE}

% "preprint" option is used for arXiv or other preprint submissions
 % \usepackage[preprint]{neurips_2025}

% to avoid loading the natbib package, add option nonatbib:
%    \usepackage[nonatbib]{neurips_2025}

\usepackage[utf8]{inputenc} % allow utf-8 input
\usepackage[T1]{fontenc}    % use 8-bit T1 fonts
\usepackage{hyperref}       % hyperlinks
\usepackage{url}            % simple URL typesetting
\usepackage{booktabs}       % professional-quality tables
\usepackage{amsfonts}       % blackboard math symbols
\usepackage{nicefrac}       % compact symbols for 1/2, etc.
\usepackage{microtype}      % microtypography
\usepackage{xcolor}         % colors
\usepackage{mathtools}
\usepackage{amsmath,amssymb}

% Note. For the workshop paper template, both \title{} and \workshoptitle{} are required, with the former indicating the paper title shown in the title and the latter indicating the workshop title displayed in the footnote. 
\title{Continuous Fused Optimal Transport}


% The \author macro works with any number of authors. There are two commands
% used to separate the names and addresses of multiple authors: \And and \AND.
%
% Using \And between authors leaves it to LaTeX to determine where to break the
% lines. Using \AND forces a line break at that point. So, if LaTeX puts 3 of 4
% authors names on the first line, and the last on the second line, try using
% \AND instead of \And before the third author name.


\author{%
  Junhyoung Chung\thanks{\href{https://junhyoung-chung.github.io/}{https://junhyoung-chung.github.io/}} \\
  Department of Statistics\\
  Seoul National University\\
  Seoul 08826, Republic of Korea \\
  \texttt{junhyoung0534@gmail.com} \\
  % examples of more authors
  % \And
  % Coauthor \\
  % Affiliation \\
  % Address \\
  % \texttt{email} \\
  % \AND
  % Coauthor \\
  % Affiliation \\
  % Address \\
  % \texttt{email} \\
  % \And
  % Coauthor \\
  % Affiliation \\
  % Address \\
  % \texttt{email} \\
  % \And
  % Coauthor \\
  % Affiliation \\
  % Address \\
  % \texttt{email} \\
}


\begin{document}


\maketitle


\begin{abstract}
  The abstract paragraph should be indented \nicefrac{1}{2}~inch (3~picas) on
  both the left- and right-hand margins. Use 10~point type, with a vertical
  spacing (leading) of 11~points.  The word \textbf{Abstract} must be centered,
  bold, and in point size 12. Two line spaces precede the abstract. The abstract
  must be limited to one paragraph.
\end{abstract}


\section{Introduction}

\section{Fused Optimal Transport}
\paragraph{Notations.} Let $(\Omega,\mathcal{F},\mathbb{P})$ be a probability space and $(S,d_S)$ be a compact metric space. A measurable map $X: \Omega \to S$ is called a random element, and its distribution is defined as $\mathbb{P}_X \coloneqq \mathbb{P} \circ X^{-1}$. In addition, we introduce a feature space $(M,d_M)$, which is also a compact metric space. Any measurable map $f : S \to M$ is called a feature function.
For two random elements $X$ and $Y$, we say a measurable map $T: S \to S$ pushes forward $\mathbb{P}_X$ to $\mathbb{P}_Y$, or simply $X$ to $Y$, if $\mathbb{P}_X(T^{-1}(A)) = \mathbb{P}_Y(A)$ for all $A \in \mathcal{B}(S)$, where $\mathcal{B}(S)$ is the Borel $\sigma$-algebra of $S$. We denote $T_{\#}\mathbb{P}_X = \mathbb{P}_Y$ or $T(X) \overset{d}{=} Y$ if $T$ pushes forward $X$ to $Y$.

\paragraph{Fused optimal transport.} For some known feature function $f$, we consider the following problem: for some $0 \leq \alpha \leq 1$,
\begin{align}
	\label{eq:fused_optimal_transport}
	\min_{T: T_{\#}\mathbb{P}_X = \mathbb{P}_Y} (1-\alpha)&\mathbb{E}_{X \sim \mathbb{P}_X} \Big[d_M(f(X),f(T(X)))\Big] \nonumber\\
	&+ \alpha \mathbb{E}_{(X,X^\prime) \sim \mathbb{P}_X \otimes \mathbb{P}_X}\Big[K_h(X,X^\prime)\left\vert d_S(X,X^\prime) - d_S(T(X),T(X^\prime)) \right\vert^2\Big] ,
\end{align}
where $K_h(\cdot,\cdot): S \times S \to [0,\infty)$ is a bounded and symmetric ($K_h(x,x^\prime) = K_h(x^\prime,x)$) kernel with a bandwidth $h > 0$. A solution for \eqref{eq:fused_optimal_transport}, denoted as $T^\ast$, is called a fused optimal transport (OT) plan.


\end{document}