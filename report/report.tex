\documentclass{article}

% if you need to pass options to natbib, use, e.g.:
%     \PassOptionsToPackage{numbers, compress}{natbib}
% before loading neurips_2025

% The authors should use one of these tracks.
% Before accepting by the NeurIPS conference, select one of the options below.
% 0. "default" for submission
% \usepackage{neurips_2025}
% the "default" option is equal to the "main" option, which is used for the Main Track with double-blind reviewing.
% 1. "main" option is used for the Main Track
%  \usepackage[main]{neurips_2025}
% 2. "position" option is used for the Position Paper Track
%  \usepackage[position]{neurips_2025}
% 3. "dandb" option is used for the Datasets & Benchmarks Track
 % \usepackage[dandb]{neurips_2025}
% 4. "creativeai" option is used for the Creative AI Track
%  \usepackage[creativeai]{neurips_2025}
% 5. "sglblindworkshop" option is used for the Workshop with single-blind reviewing
 % \usepackage[sglblindworkshop]{neurips_2025}
% 6. "dblblindworkshop" option is used for the Workshop with double-blind reviewing
%  \usepackage[dblblindworkshop]{neurips_2025}

% After being accepted, the authors should add "final" behind the track to compile a camera-ready version.
% 1. Main Track
\usepackage[main, final]{neurips_2025}
% 2. Position Paper Track
%  \usepackage[position, final]{neurips_2025}
% 3. Datasets & Benchmarks Track
 % \usepackage[dandb, final]{neurips_2025}
% 4. Creative AI Track
%  \usepackage[creativeai, final]{neurips_2025}
% 5. Workshop with single-blind reviewing
%  \usepackage[sglblindworkshop, final]{neurips_2025}
% 6. Workshop with double-blind reviewing
%  \usepackage[dblblindworkshop, final]{neurips_2025}
% Note. For the workshop paper template, both \title{} and \workshoptitle{} are required, with the former indicating the paper title shown in the title and the latter indicating the workshop title displayed in the footnote.
% For workshops (5., 6.), the authors should add the name of the workshop, "\workshoptitle" command is used to set the workshop title.
% \workshoptitle{WORKSHOP TITLE}

% "preprint" option is used for arXiv or other preprint submissions
 % \usepackage[preprint]{neurips_2025}

% to avoid loading the natbib package, add option nonatbib:
%    \usepackage[nonatbib]{neurips_2025}

\usepackage[utf8]{inputenc} % allow utf-8 input
\usepackage[T1]{fontenc}    % use 8-bit T1 fonts
\usepackage{hyperref}       % hyperlinks
\usepackage{url}            % simple URL typesetting
\usepackage{booktabs}       % professional-quality tables
\usepackage{amsfonts}       % blackboard math symbols
\usepackage{nicefrac}       % compact symbols for 1/2, etc.
\usepackage{microtype}      % microtypography
\usepackage{xcolor}         % colors
\usepackage{mathtools}
\usepackage{amsmath,amssymb,amsthm}
\usepackage{cleveref}

\newtheorem{theorem}{Theorem}
\newtheorem{lemma}{Lemma} 
\newtheorem{proposition}{Proposition} 
\newtheorem{corollary}{Corollary}
\newtheorem{definition}{Definition} 
\newtheorem{example}{Example} 
\newtheorem{assumption}{Assumption}
\newtheorem{claim}{Claim}
\newtheorem{remark}{Remark}
\newtheorem{note}{Note} 
\newtheorem{question}{Question}

\crefname{theorem}{Theorem}{Theorems}
\Crefname{theorem}{Theorem}{Theorems}

\crefname{lemma}{Lemma}{Lemmas}
\Crefname{lemma}{Lemma}{Lemmas}

\crefname{proposition}{Proposition}{Propositions}
\Crefname{proposition}{Proposition}{Propositions}

\crefname{corollary}{Corollary}{Corollaries}
\Crefname{corollary}{Corollary}{Corollaries}

\crefname{definition}{Definition}{Definitions}
\Crefname{definition}{Definition}{Definitions}

\crefname{example}{Example}{Examples}
\Crefname{example}{Example}{Examples}

\crefname{assumption}{Assumption}{Assumptions}
\Crefname{assumption}{Assumption}{Assumptions}

\crefname{claim}{Claim}{Claims}
\Crefname{claim}{Claim}{Claims}

\crefname{remark}{Remark}{Remarks}
\Crefname{remark}{Remark}{Remarks}

\crefname{note}{Note}{Notes}
\Crefname{note}{Note}{Notes}

\crefname{question}{Question}{Questions}
\Crefname{question}{Question}{Questions}

% Note. For the workshop paper template, both \title{} and \workshoptitle{} are required, with the former indicating the paper title shown in the title and the latter indicating the workshop title displayed in the footnote. 
\title{Continuous Fused Optimal Transport}


% The \author macro works with any number of authors. There are two commands
% used to separate the names and addresses of multiple authors: \And and \AND.
%
% Using \And between authors leaves it to LaTeX to determine where to break the
% lines. Using \AND forces a line break at that point. So, if LaTeX puts 3 of 4
% authors names on the first line, and the last on the second line, try using
% \AND instead of \And before the third author name.


\author{%
  Junhyoung Chung\thanks{\href{https://junhyoung-chung.github.io/}{https://junhyoung-chung.github.io/}} \\
  Department of Statistics\\
  Seoul National University\\
  Seoul 08826, Republic of Korea \\
  \texttt{junhyoung0534@gmail.com} \\
  % examples of more authors
  % \And
  % Coauthor \\
  % Affiliation \\
  % Address \\
  % \texttt{email} \\
  % \AND
  % Coauthor \\
  % Affiliation \\
  % Address \\
  % \texttt{email} \\
  % \And
  % Coauthor \\
  % Affiliation \\
  % Address \\
  % \texttt{email} \\
  % \And
  % Coauthor \\
  % Affiliation \\
  % Address \\
  % \texttt{email} \\
}


\begin{document}


\maketitle


\begin{abstract}
  The abstract paragraph should be indented \nicefrac{1}{2}~inch (3~picas) on
  both the left- and right-hand margins. Use 10~point type, with a vertical
  spacing (leading) of 11~points.  The word \textbf{Abstract} must be centered,
  bold, and in point size 12. Two line spaces precede the abstract. The abstract
  must be limited to one paragraph.
\end{abstract}


\section{Introduction}

\section{Fused Optimal Transport}
\paragraph{Notations.} Let $(\Omega,\mathcal{F},\mathbb{P})$ be a probability space and $(S,d_S)$ be a compact metric space. A measurable map $X: \Omega \to S$ is called a random element, and its distribution is defined as $\mathbb{P}_X \coloneqq \mathbb{P} \circ X^{-1}$. In addition, we introduce a feature space $M \subset \mathbb{R}^d$, which is also a compact region. Any measurable map $f : S \to M$ is called a feature function.
For two random elements $X$ and $Y$, we say a measurable map $T: S \to S$ pushes forward $\mathbb{P}_X$ to $\mathbb{P}_Y$, or simply $X$ to $Y$, if $\mathbb{P}_X(T^{-1}(A)) = \mathbb{P}_Y(A)$ for all $A \in \mathcal{B}(S)$, where $\mathcal{B}(S)$ is the Borel $\sigma$-algebra of $S$. We denote $T_{\#}\mathbb{P}_X = \mathbb{P}_Y$ if $T$ pushes forward $X$ to $Y$.

\paragraph{Fused optimal transport.} For some feature function $f$, we consider the following problem: for $0 \leq \alpha \leq 1$,
\begin{align}
	\label{eq:T-fused-optimal-transport}
	\min_{T: T_{\#}\mathbb{P}_X = \mathbb{P}_Y} (1-\alpha)&\mathbb{E}_{X \sim \mathbb{P}_X} \Big[ \| f(X) - f(T(X)) \|_2^2 \Big] \nonumber\\
	&+ \alpha \mathbb{E}_{(X,X^\prime) \sim \mathbb{P}_X \otimes \mathbb{P}_X}\Big[K_h(X,X^\prime)\left\vert d_S(X,X^\prime) - d_S(T(X),T(X^\prime)) \right\vert^2\Big] ,
\end{align}
where $\|\cdot\|_2$ is a Euclidean norm, and $K_h(\cdot,\cdot): S \times S \to [0,\infty)$ is a bounded symmetric kernel with a bandwidth $h > 0$. Throughout this study, we assume that there exist solutions for \eqref{eq:T-fused-optimal-transport}.
\begin{assumption}
	\label{ass:existence}
	For all $0 \leq \alpha \leq 1$, there exists a non-empty solution set $\mathcal{S}_\alpha$ for \eqref{eq:T-fused-optimal-transport}.
\end{assumption}

%%% Conventional OT %%%
\eqref{eq:T-fused-optimal-transport} reduces to a conventional OT problem when $\alpha = 0$, which have been widely discussed across several literature. In this case, there exists a $\mathbb{P}_X$-a.e. uniqueness solution, denoted as $T^\ast$, which is guaranteed by the Brenier theorem under some regular assumptions.

\begin{assumption}
	\label{ass:feature-function}
	A feature function $f$ is one-to-one and continuous.
\end{assumption}

\begin{assumption}
	\label{ass:absolute-continuity}
	%The distribution of $f(X)$, $\mathbb{P}_{f(X)} = f_{\#}\mathbb{P}_X$, is dominated by the Lebesgue measure.
	The distribution of $f(X)$ is dominated by the Lebesgue measure.
\end{assumption}

\begin{assumption}
	\label{ass:finite-second-moment}
	$f(X)$ and $f(Y)$ have finite second moments, i.e., $\mathbb{E}[\|f(X)\|_2^2], \mathbb{E}[\|f(Y)\|_2^2] < \infty$.
\end{assumption}

\begin{lemma}
	\label{lem:brenier-theorem}
	Consider a fused optimal transport problem in \eqref{eq:T-fused-optimal-transport} with $\alpha = 0$. Under \cref{ass:feature-function,ass:absolute-continuity,ass:finite-second-moment}, there exists a $\mathbb{P}_X$-a.e. unique solution $T^\ast$.
\end{lemma}
\cref{lem:brenier-theorem} illustrates the $\mathbb{P}_X$-a.e. uniqueness of the solution for the special case of \eqref{eq:T-fused-optimal-transport} with $\alpha = 0$, which is the simple application of the Brenier theorem.

%%% Fused OT %%%
In contrast, global solutions for \eqref{eq:T-fused-optimal-transport} with $\alpha > 0$ may not be unique due to the non-convexity of the second term. However, we can still guarantee that there exists a $T^\ast_\alpha \in \mathcal{S}_\alpha$ near by $T^\ast$ under some required assumptions. To this end, we first introduce the following alternative problem for \eqref{eq:T-fused-optimal-transport}, which is essentially equivalent: for $Z \coloneqq f(X)$,
\begin{align}
	\label{eq:U-fused-optimal-transport}
	\min_{U: U_{\#}\mathbb{P}_Z = f_{\#}\mathbb{P}_Y} (1-\alpha)\mathbb{E}_{Z \sim \mathbb{P}_Z} \Big[ \| Z - U(Z) \|_2^2 \Big] + \alpha R(U) ,
\end{align}
where
\begin{align}
	\label{eq:U-risk}
	R(U) \coloneqq \mathbb{E}_{(Z,Z^\prime) \sim \mathbb{P}_Z \otimes \mathbb{P}_Z}\Big[K_h^f(Z,Z^\prime)\left\vert d_S(g(Z),g(Z^\prime)) - d_S(g(U(Z)),g(U(Z^\prime))) \right\vert^2\Big] .
\end{align}
Here $K_h^f(z,z^\prime) \coloneqq K_h(g(z),g(z^\prime))$, and $g \coloneqq f^{-1}$, where $f^{-1}$ is defined on $f(S)$.
	

%However, the following theorem confirms that there exists a $T^\ast_\alpha \in \mathcal{S}_\alpha$ near by $T^\ast$. 


%%%%%%%%%%%%%%%%%%%%%%%%%%%%%%%%%%%%%%%%%%%%%%%%%%%          %%%%%%%%%%%%%%%%%%%%%%%%%%%%%%%%%%%%%%%%%%%%%%%%%%%
%%%%%%%%%%%%%%%%%%%%%%%%%%%%%%%%%%%%%%%%%%%%%%%%%%% Appendix %%%%%%%%%%%%%%%%%%%%%%%%%%%%%%%%%%%%%%%%%%%%%%%%%%%
%%%%%%%%%%%%%%%%%%%%%%%%%%%%%%%%%%%%%%%%%%%%%%%%%%%          %%%%%%%%%%%%%%%%%%%%%%%%%%%%%%%%%%%%%%%%%%%%%%%%%%%

\appendix
\section{Appendix}

\subsection{Proof for \cref{lem:brenier-theorem}}

\begin{proof}
	Since $\alpha = 0$, \eqref{eq:T-fused-optimal-transport} boils down to the following optimization problem:
	\begin{align}
		\label{eq:T-ot}
		\min_{T: T_{\#}\mathbb{P}_X = \mathbb{P}_Y} \mathbb{E}\Big[\|f(X) - f(T(X))\|_2^2\Big] .
	\end{align}
	Define $Z \coloneqq f(X)$, whose distribution $\mathbb{P}_Z$ is equal to $f_{\#}\mathbb{P}_X$. In addition, by \cref{ass:feature-function}, we can define $g \coloneqq f^{-1}$, where $f^{-1}$ is defined on $f(S)$. Then, the above problem is equivalent to
	\begin{align}
		\label{eq:U-ot}
		%\min_{T: T_{\#}\mathbb{P}_X = \mathbb{P}_Y} \mathbb{E}\Big[\|Z - (f \circ T \circ g)(Z)\|_2^2\Big] = \min_{U: U_{\#}\mathbb{P}_Z = f_{\#}\mathbb{P}_Y} \mathbb{E}\Big[\|Z - U(Z)\|_2^2\Big] .
		\min_{U: U_{\#}\mathbb{P}_Z = f_{\#}\mathbb{P}_Y} \mathbb{E}\Big[\|Z - U(Z)\|_2^2\Big] .
	\end{align}
	This is because $(f \circ T \circ g)_{\#}\mathbb{P}_Z = f_{\#}\mathbb{P}_Y$ when $T_{\#}\mathbb{P}_X = \mathbb{P}_Y$. Now, the Brenier theorem guarantees that under \cref{ass:absolute-continuity,ass:finite-second-moment}, there exists a $\mathbb{P}_Z$-a.e. unique solution $U^\ast$ for \eqref{eq:U-ot}.
	
	Let $T^\ast \coloneqq g \circ U^\ast \circ f$. Observe that $T^\ast_{\#}\mathbb{P}_X = g_{\#}(U^\ast_{\#}\mathbb{P}_Z) = \mathbb{P}_Y$ and that $f(T^\ast(X)) = U^\ast(Z)$, which implies that $T^\ast$ is a solution for \eqref{eq:T-ot}.
	
	What remains is to show that $T^\ast$ is $\mathbb{P}_X$-a.e. unique. Suppose that there exists another solution $\tilde{T}^\ast$ such that $\mathbb{P}(T^\ast(X) \neq \tilde{T}^\ast(X)) > 0$. Then,
	\begin{align*}
		\mathbb{P}(T^\ast(X) \neq \tilde{T}^\ast(X)) = \mathbb{P}(g(U^\ast(Z)) \neq (\tilde{T}^\ast \circ g)(Z)) = \mathbb{P}(U^\ast(Z) \neq (f \circ \tilde{T}^\ast \circ g)(Z)) > 0 .
	\end{align*}
	Considering that $\tilde{U}^\ast \coloneqq f \circ \tilde{T}^\ast \circ g$ is also a solution for \eqref{eq:U-ot}, this contradicts to the fact that $U^\ast$ is $\mathbb{P}_Z$-a.e. unique.
\end{proof}

\end{document}